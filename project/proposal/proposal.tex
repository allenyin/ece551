\documentclass[12pt]{article}
\usepackage{fullpage}
\usepackage{multirow}

%NOTE: This proposal is intended as 
% (1) A template for your document
% (2) An example of the writing.
% It is not intended as an example of the project's scope:
% The project proposed in this sample is purposefully WAY
% beyond the scope of the class.

\begin{document}
\begin{center}
% Project Title
{\Large Micro-architectural Processor Simulator}

\vspace*{20pt}
% List your group members' names and NetIds
\begin{tabular}{lc}
Andrew Hilton  & adh39 \\
Tami Lehman & ts179 \\
Patrick Huang    & zq28 \\
\end{tabular}
\end{center}
%Give a brief (one paragraph) summary of the project.

For our project, we will develop a micro-architectural processor
simulator capable of simulating the x86\_64 (user space) ISA.  This
processor will load and execute user-level binaries.  We will emulate
system calls functionally in the simulator---we do not intend to 
implement full system simulation.


\section{Functional Requirements}
% Describe the functional requirements.
% This is what you would need to do for a mid-B to a low-A
% Be *as specific as possible here*.  If you specifically 
% want to make clear that certain functionality is outside
% the scope of what you propose, please be sure to do so.


\begin{description}
\item[Memory] We will implement a functional model of physical memory,
  as well as giving each process its own virtual address space and
  translation.  As we will not simulate an operating system, the
  virtual to physical translations for a program will be fixed once
  they are created.
\item[ELF binary loading]  We will implement an ELF loader capable of
  reading a statically linked ELF binary into our simulated memory. 
  We reserve loading dynamically linked binaries for a stretch goal.
\item[Instruction Decoding] We will implement an x86\_64 instruction
  decoder, capable of decoding the subset of x86\_64 instructions
  which appear in the SPEC2006 benchmark suite.  Note that we do not
  intend to implement \emph{all} x86\_64 instructions, as there are
  quite a large number of them.
\item[Micro-op cracking] We will crack the x86\_64 micro-ops into a
  simple RISC-style micro-ISA.
\item[Micro-op execution] We will provide a functional model capable
  of executing our micro-ops (and thus the instructions we can
  decode).
\end{description}

\section{Stretch Goals}
% Once you have fulfilled your basic requirements, what do you
% see as the more ambitious steps to take you into the high-A range?
% Its good if you can be precise, but, for these it is acceptable
% to be a bit more vague.

\begin{description}
\item[Simple timing model] Our firs stretch goal will be to provide a
  simple 5 stage in-order scalar timing model, with a 2 level cache
  hierarchy.  The timing model will report IPC, but no other stats.
\item[Branch Prediction] Our second stretch goal will be to implement a
  branch predictor for our timing model.
\item[Super Scalar] Our third stretch goal will be to add an option to
  increase the super scalar width of the processor.
\item[Dynamically linked ELF loading] Our fourth stretch goal will be
  to support loading of dynamically linked ELF binaries.
  simulations.
\end{description}


\section{Software Design Plan}
% In this section, you should lay out a rough plan to build your 
% software.  How do you split the work amongst a team?  What
% pieces are you going to build first?  Can you test along the way?
% Are there any other things you need to build which are not
% part of your ``deliverables''? (e.g., to help test)
% What is your project timeline?  How many person-hours of work
% do you expect for each task?
%
% The key thing here is that I want to see that you have thought
% about what it will take to accomplish your goal, and planned
% on a reasonably sized project
%
% I should feel like you have a plan to at least get through your
% functional requirements within the timeline of this project
% (and picking a project substanative enough for a final project).
%
% A good rule of thumb is about 30 hours per person, since
% you no longer have  videos to watch and this project spans 3.5 weeks

Our first step will be to get the binary image loaded into memory.  We
see two discrete tasks in this step: creating the memory model, and
writing the ELF loader.  We plan to split these tasks between two
group members.  As we build this, we will create functionality in
our memory model to dump its contents, and use that to test the ELF
loader functionality.

At the same time that is being developed, our third group member will
begin making the instruction decoder.  We plan to build a tool to let
us write a convenient description of the instruction encoding, and
generate an FSM for the decoder.  This tool will also allow us to
specify the micro-op crackings of each instruction. As this piece is
developed, we will write down the encodings of many instructions and
use them as test cases.  We expect to test this by examining the FSM
(C code) output by the tool.

Once the memory model and/or ELF loader are complete, those teammates
will transition to working on the execution capabilities. At this
point, we will be able to test by loading and actual binaries
(starting simple, and working to more complex ones).

Once we are to this phase, it will mostly be a matter of running
programs until we find an instruction we cannot handle, adding
that functionality, and running again.

\begin{table}
\begin{center}
\begin{tabular}{|c|c|c|c|c|c|c|}
\hline
\multirow{2}{*}{Task}      &  \multicolumn{3}{|c|}{Expected Person-Hours}    &  \multirow{2}{*}{Start} & \multirow{2}{*}{End} & \multirow{2}{*}{Person} \\
                           &  Implement     & Test       & Debug            &                         &                      &                         \\
\hline
\hline
  Memory Model             &   2            & 3          & 3                & Nov 3                   & Nov 9                & Tami  \\
  ELF Loader               &   4            & 6          & 4                & Nov 3                   & Nov 12               & Patrick \\
  Decode/Crack             &   4            & 4          & 4                & Nov 3                   & Nov 15               & Drew \\
  Execution (75\%)         &   6            & 6          & 3                & Nov 9                   & Nov 21               & Tami \\
  Execution (25\%)         &   2            & 2          & 1                & Nov 9                   & Nov 21               & Patrick \\
\hline
 \multicolumn{7}{|c|}{Stretch Goals} \\
\hline
Simple Timing Model        &   3           & 3         &   4                & Nov 15                  & Nov 21               & Drew \\
Branch Prediction          &   2           & 2         &   3                & Nov 21                  & Nov 27               & Tami \\
Super Scalar               &   3           & 3         &   3                & Nov 21                  & Nov 27               & Drew \\
Dynamicly linked ELF       &   4           & 3         &   3                & Nov 21                  & Nov 27               & Patrick \\

\hline
\end{tabular}
\caption{Summary of work breakdown and time expectations.}
\label{Table:planSummary}
\end{center}
\end{table}

Table~\ref{Table:planSummary} sumarizes our expected time breakdown by task, as well as who is planning to do each task, and what our exected schedule of starting/completing the tasks is.


\section{Specific Expertise}
% Lay out the specific technical expertise (beyond just programming)
% that your group needs to succeed.  If you have that expertise
% already or if you have/know of the resources you need, point that
% out.  If you lack information and will need me to help you find it,
% this is the time to let me know.
% 
% That last part is important, so I'll say it again:
% If there is specific technical expertise that you might lack,
% tell me NOW.  If you propose (complex artificial intelligence thing
% I know nothing about), don't know to me a week before the deadline
% and tell me that you don't know anything about AI...

This project requires significant expertise in processor
micro-architecture, as well as the x86\_64 ISA.  Our group members
have much expertise in this area, so we do not expect many problems.

We do not remember the details of the ELF format (especially for
dynamically linked ELF binaries), but expect to be able to find that 
in online documentation.

We have an ISA manual for x86\_64, so we can find whatever details
we need there as we implement the instructions.

\end{document}
